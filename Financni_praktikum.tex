\documentclass[a4paper,12pt]{article}
\usepackage[slovene]{babel}
\usepackage[utf8]{inputenc}
\usepackage{mathtools}
\usepackage{amsfonts}
\usepackage{graphicx} %za slike
\usepackage[none]{hyphenat} %da se besede ne delijo
\setlength{\parindent}{0mm} %da ne bo zamika na začetku odstavkov
%\newcommand{\pojem}[1]{\textsc{#1}} 
%\usepackage{textcomp} %sem vklučila zaradi  x pri 4x4 
\usepackage{enumitem}
\usepackage{gensymb}


\begin{document}


\thispagestyle{empty}%
   \begin{center}
    {\large\sc Univerza v Ljubljani\\%
      Fakulteta za matematiko in fiziko}\\%
      {\large  Univerzitetni študijski program\\ Finančna matematika\par}%
    \vskip 10em%
    {\vskip 3em \textsc{FINANČNI PRAKTIKUM\\ \large HORIZONTALNO PREKRIVANJE POLIGONOV}\\\par}%
    \vfill\null%[5mm]     
    {\large \textsc{Avtorja}: Luka Bizjak, Lucija Udir\par}%  
    %{\large \textsc{Mentor}: doc.\  izred. prof. dr. Janez Bernik,\\ prof. dr. Tomaž Košir\par}%
    {\vskip 2em \large Ljubljana, 2021 \par}%
\end{center}

\newpage
\section{Navodilo}
Generiraj moder in rdeč konveksen poligon v ravnini ter zapiši linearni program, ki poišče potreben minimalni horizontalni premik enega izmed teh dveh poligonov, da poligona postaneta disjunktna. 

\section{Opis problema in načrt za nadalnje delo}
%Poligona sta disjunktna, če je presek teh dveh likov prazna množica. 

Zgornji problem bova reševala z linearnim programiranjem, ki nam omogoča poiskati optimalno (maksimalno ali minimalno) vrednost izbranih odvisnih spremenljivk,
 ki zadoščajo določenim omejitvam. Kriterijska funkcija, ki jo maksimiziramo ali minimiziramo, je linearna.  Pri reševanju naloge bova uporabljala programski jezik \textbf{Python}. \\
 
 
 Konstruirala bova naključna konveksna poligona in izračunala njuno horizontalno prekrivanje. Konveksen poligon je enostaven mnogokotnik, katerega notranjost je konveksna množica.
 To pomeni, da vsaka daljica med dvema točkama, ki ležita v notranjosti ali na robu poligona, tudi leži v njem. \\ 


 Za reševanje bova potrebovala naslednje funkcije:
\begin{itemize}
\item Funkcijo, ki preveri ali je dani poligon konveksen. Torej bo funkcija preverila ali je vsak notranji kot manjši ali enak $180$ \degree. 
\item Funkcijo, ki bo zgenerirala konveksen poligon in funkcijo, ki ga bo izrisala na koordinatnem sistemu.
\item Funkcijo za premikanje poligona v smeri abcisne osi, ki bo $x$ koordinatam poligona prištela oziroma odštela dano vrednost. 
\item Funkcijo, ki bo poiskala potreben minimalni premik rdečega poligona v smeri abcisne osi, da rdeč in moder poligon postaneta disjunktna. To bova izračunala s pomočjo iskanja minimalne absolutne vrednosti ničel 
funkcije \texttt{f(x, RdecPoligon, ModerPoligon)}, ki premakne rdeč poligon za dani $x$, izračuna presek ter oddaljenost premaknjenega poligona in modrega poligona. Funkcija 
bo vrnila razliko med presekom in razdaljo. Ko bosta poligona disjunktna, bo ploščina njunega preseka enaka $0$ . Razdalja med njima pa nenegativna. Z večanjem premika rdečega poligona,
 bo vrednost te funkcije vedno bolj negativna. V primeru, ko ploščina preseka teh dveh poligonov ne bo enaka $0$, bo razdalja enaka $0$. Tako bo vrednost funkcije v tem primeru pozitivna. 
\item Funkcija, ki bo s poočjo Newtonove metode poiskala najmanjšo (po absolutni vrednosti) ničlo prejšne funkcije. 
\end{itemize}

%Linearni program:
%Podatki: Koordinate dveh konveksnih poligonov in njunega preseka 
%Omejitve: ploscina preseka >= 0, konveksnost
%Ciljna funkcija: Poiskati moramo minimalno oddaljenost ničel funkcije f(A,d) = A - d od koordinatnega izhodišča. 


%\section{LP program}
%
%Za iskanje minimalnega vodoravnega premika dveh poligonov, da postaneta disjunktna bova uporabila sledeč  LP:\\
%
%Vhodni podatki:
%\begin{itemize}
%%\item{$n$ \dots število oglišč poligona; $n \in \mathbb{N}$}
%\item{$A$ \dots ploscina konveksnega preseka modrega in rdečega poligona}
%\item{$d$ \dots oddaljenost modrega poligona od rdečega}
%%\item{$x_{i}$ \dots $x$ koordinata $i$-te točke}
%%\item{$y_{i}$ \dots $y$ koordinata $i$-te točke}
%\end{itemize}
%
%Spremenljivke:
%\begin{itemize}
%\item{$x_{i}= 1$, če je ničla funkcije $f = A - D$ } , sicer je 0
%\end{itemize}
%
%%Točka $(x_i,y_i)$ je v kvadratu generiranim z $(x_j, y_j)$, če veljajo sledeči pogoji:
%%$$x_j \le x_i \le x_j+1$$
%%$$y_j \le y_i \le y_j+1$$
%
%Iščemo\\
%$$\max_{i \in Z} \sum_{j\in Z} x_{ij} $$
%% tu poiščemo največje število presekov kvadratov
%pri pogojih:
%$$x_{ij} \in \{0,1\} \quad\forall i,j \in \mathbb{N}\\$$



\end{document}

koordinate so cela števila
liki ležijo v kvadratu omejenem s (0,0) in (C,C)
število stranic je določeno z N

X = [0, ..., c]
Y = [0, ..., c]
sortiraj X in Y in določi min in max